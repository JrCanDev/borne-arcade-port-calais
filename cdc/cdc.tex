\documentclass[11pt]{article}

\usepackage{a4wide}

\usepackage[utf8]{inputenc}
\usepackage[francais]{babel}

\usepackage{fourier}

\usepackage{MnSymbol,wasysym}

\usepackage{url}

\title{Cahier des charges pour le projet de la borne au port de Calais\\
\large Partie logicielle traitée par l'IUT de Calais}
\author{Rémi Synave}
\date{v0.2 du \today}

\begin{document}

\maketitle

\danger \textbf{Ce document ne concerne que la partie développement logiciel du projet. Le projet est présenté dans sa totalité mais les besoins fonctionnels exprimés dans ce document ne concernent que le développement confié à l'IUT.}

\section{Descriptif du projet}
\subsection{Présentation des acteurs}
Le projet met en collaboration plusieurs entités :
\begin{itemize}
\item La ville de Calais au travers du Sas Coluche pour la création du contenu : rédaction du texte et prise de photos.
\item Le port de Boulogne Calais pour l'accueil du produit fini dans ses locaux.
\item La micro entreprise ?? (TODO) du lycée Courbertin pour la fabrication du meuble (appelé \textbf{borne} dans la suite du document).
\item Le département informatique de l'IUT de Calais et l'association JrCanDev.\\

\end{itemize}
Le projet est conduit par un ensemble de personnes appartenant aux entités précédentes.\\~

Le \textbf{Sas Coluche} est une structure de la ville de Calais qui a deux activités principales : centre de loisirs pour les vacances scolaires et une structure qui aide les jeunes en situation de difficulté. Ils vont dans les quartiers défavorisés, font du repérage sur les réseaux sociaux et autres pour essayer de remettre les jeunes décrocheurs sur le chemin de l'emploi ou de la formation au travers de projets. Des projets se concrétisent par une embauche de jeunes avec le partenaire ou une reprise d'étude. Le Sas Coluche essaie de récolter des fonds auprès des différents partenaires, de la mairie, etc.\\
Le Sas Coluche se trouve Place Coluche à Calais.\\~

Le \textbf{port boulogne Calais} accueille les voyageurs en partance vers l'Angleterre ou qui en reviennent mais également le transport de marchandise sur camion. La borne sera placée dans le bâtiment SEPD qui se trouve en zone public. Cet endroit est également un lieu de rassemblement avant ou après la traversée.\\~

La micro entreprise\textbf{??}. Elle a été créée dans les locaux et par les personnes du lycée Coubertin. TODO\\~

Le \textbf{département informatique de l'IUT de Calais} forme tous les ans des étudiants au développement de logiciel et développement web entre autres. Cet acteur fera travailler les étudiants de niveaux différents (BUT1 à BUT3) sur ce projet au travers de projets pédagogiques et scolaires durant l'année.\\
\textbf{JrCanDev} est une association réunissant enseignant et étudiant et qui propose des services d'hébergement, de développement, d'impression 3D, etc. Des étudiants travailleront sur le logiciel en parallèle des projets suivis lors de leurs cours.\\\url{https://www.jrcandev.netlib.re/}\\~
 
 \subsection{Le sujet}
Précédemment, des jeunes encadrés par le Sas Coluche ont créé une \textbf{application ipad} qui a été proposée comme support à la visite des officiels lors de l'inauguration du port de Calais. Les jeunes ont créé le contenu. Ce projet a permis au Sas Coluche de faire travailler les jeunes sur le Français mais aussi de leur faire reprendre contact avec le monde extérieur et de lier des contacts.\\ L'entreprise partenaire était Bouygues (l'entreprise qui a construit le port).\\

L'application consiste en une suite de "pages" qui présentent une partie du chantier, un bâtiment, un moment d'histoire, ou tout autre descriptif. Les pages présentées peuvent également contenir un quizz ou une vue 3D. La partie développement informatique de l'application a été sous-traitée. Le Sas Coluche n'a pas accès au code source de l'application. Le résultat est très satisfaisant.\\

Le Sas Coluche souhaiterait continuer dans cette même voie en créant une seconde application traitant d'un autre sujet. Le coût de développement d'une telle application est trop élevé pour que cette initiative puisse être renouvellée sans une aide financière extérieure. De plus, l'application existante peut être améliorée sur deux points : l'application demande une connexion internet pour fonctionner et n'est compatible que pour les ipad.\\

C'est dans ce cadre que les étudiants de l'IUT de Calais interviendront.

\section{Solution proposée}
\subsection{Solution classique}
L'application sera un ensemble de pages web affichée en plein écran pour simuler une application tablette classique. Les pages web seront munis de boutons \verb@précédent@ et \verb@suivant@ qui permettront de naviguer de page en page. Les pages devront être proposées dans deux langues, français et anglais, et il sera être simple de passer de l'une à l'autre en cliquant sur un simple drapeau représentant le pays. Les pages feront apparaitre les différents partenaires du projet sous forme de petits logos en bas de page.\\~

Cette solution nécessite un développement web classique et mène à un ensemble de pages web fixes et donc une application fixe non évolutive.\\~

Cette application fixe proposée répond à la demande : les jeunes du Sas Coluche générent le contenu qui est mis en forme par des étudiants de l'IUT sous forme de pages web qui s'enchainent. Ces pages sont déposées sur l'ordinateur muni de l'écran tactile se trouvant dans la borne.
%L'ordinateur doit être configuré pour :
%\begin{itemize}
%\item être utilisé en mode tablette pour utiliser l'écran virtuel si nécessaire,
%\item démarrer en ouvrant la première page web créée dans un navigateur en mode plein écran.
%\end{itemize}~\\

\danger Cet ensemble de pages web affichées sur l'écran tactile et simulant l'application ipad existante sera appelé \textbf{application tablette} dans la suite du document. L'installation de cette \textbf{application tablette} consistera simplement en la copie des pages web dans l'ordinateur de la borne.\\

L'application tablette générée contiendra toutes les informations pour un affichage sur la borne. Ainsi, aucune connexion internet ne sera requise lors de l'utilisation de la borne. De plus, cette solution est totalement portable dans le sens où les pages web sont standards et peuvent être affichés sur tout ordinateur disposant d'un navigateur.\\~

\subsection{Solution améliorée}
Il est probable que cette application tablette ait besoin d'évoluer dans le temps pour, ajouter ou modifier des informations, pour corriger des erreurs ou même changer complètement le sujet traité. L'IUT propose alors de créer un logiciel qui faciliterait la création des pages web pour que tout le monde puisse générer une nouvelle application tablette facilement et ainsi la mettre à jour simplement sur la borne sans avoir recours à un développeur.\\~

Le logiciel final attendu doit permettre de générer l'application tablette (un site web complet) respectant les spécifications données ci-dessus. Afin de faciliter la création des différentes pages, le logiciel doit proposer un ensemble de modèles de page prérempli/préformaté à l'utilisateur. L'utilisateur ne devra remplir que des champs pour inclure son texte ou des images.\\~

Les jeunes du Sas Coluche pourront créer alors créer l'application tablette par eux-mêmes.

\section{Contraintes}
\begin{enumerate}
\item Le logiciel permettant de générer des applications tablette sera utilisée par des personnes n'ayant pas forcément de compétence en développement. Il est donc essentiel que l'application soit intuitive et simple à utiliser. L'utilisateur type de ce logiciel est un jeune du Sas Coluche ou une y personne y travaillant.
\item Le logiciel devra être fourni avec un manuel utilisateur clair pour la partie utilisation du logiciel ainsi que contenir une partie décrivant la procédure pour "installer" l'application tablette dans la borne.
\item Le logiciel permettant de générer l'application tablette devra être utilisé sur un ordinateur tiers (pas sur l'ordinateur de la borne).
\item L'ordinateur de la borne devra être un ordinateur classique avec écran FULL HD (1920x1080) tactile. Cet ordinateur devra être configuré de façon à ce qu'il démarre en mode tablette et en exécutant un navigateur ouvrant la page d'accueil de l'application tablette (emplacement à définir en fin de projet).
\end{enumerate}


\section{Besoins / Fonctionnalités (partie à détailler)}
\begin{enumerate}
\item Définir la charte graphique qui sera utilisée sur toutes les pages.
\item Créer des modèles de page préremplis : les boutons précédent et suivant doivent être placés, la charte graphique respectée et des champs de texte/image/formulaire placé mais non rempli. Les logos des différents partenaires doivent apparaitre sur les pages dans le footer.
\item Le logiciel doit générer un ensemble de pages web utilisant html/css et javascript pour gérer les formulaires.
\item Cet ensemble de pages web doit se trouver dans un répertoire choisi ou, au moins, facilement identifiable.
\item Les pages doivent être en plein écran (sans ascenceur - pas de scroll nécessaire) pour du full HD (1920x1080).
\item Le logiciel doit embarquer un système de projet qu'il est possible de sauvegarder, ouvrir, etc.
\item Lorsque l'on crée un nouveau projet, il doit être possible de créer un projet mono ou multilingue (fr/en). Ce choix est fait lors de la création du projet et ne doit plus être modifiable ensuite.
\item Lorsqu'une nouvelle page est créée, le logiciel doit demander à l'utilisateur quel modèle doit être utilisé.
\item Une page est créée suivant un modèle prérempli. La finalisation de la page se fait en remplissant des champs d'informations qui doivent être remplis par l'utilisateur. Si du texte doit être entré et que le projet est multilingue alors il faut absolument que toutes les langues soient renseignée.
\item Ajouter une nouvelle page doit automatiquement créer le lien suivant sur le bouton de la page courante et le lien précédent du bouton de la page qui sera créée.
\item Le bouton précédent de la première page doit être désactivé.
\item le bouton suivant de la dernière page doit renvoyer vers la première page.
\item Les pages web doivent pouvoir intégrer des images 360.
\item Les pages web doivent pouvoir intégrer des formulaires dans les pages (N) et ainsi avoir deux pages suivantes (N+1) possibles (le bouton suivant doit être désactivé). Le traitement du formulaire se fera en javascript pour rediriger vers la bonne page (N+1) suivante. Le bouton précédent des pages vers lesquelles on est redirigé mènera à la page du formulaire. Le bouton précédent de la page encore suivante (N+2) redirigera également vers le formulaire.
\item Il faudrait avoir, dans le logiciel, une vision rapide du site complet, une sorte de résumé du site sous forme de schéma ou de liste/frise chronologique.
\item Le passage de la version française à la version anglaise (ou inversement) doit pouvoir se faire à tout moment et facilement grâce à un lien sous forme de drapeau en haut de la page.
\item Un manuel d'installation si le logiciel a besoin d'une installation spéciale.
\item Un manuel utilisateur qui explique comment utiliser le logiciel et qui contient également une partie "comment transferer le site web généré sur l'ordinateur de la borne".
\item Faire en sorte que les pages générées puissent être affichées sur des écrans de résolutions différentes. (utiliser des tailles et placements relatifs plutôt qu'absolu).
\item À tester sur tous les navigateurs possibles - Utiliser du HTML simple pour éviter tout problème de compatibilité avec les navigateurs.
\end{enumerate}
Tous les items sont prioritaires et obligatoires sauf les deux derniers qui permettraient d'être serein en cas de changement obligatoire de matériel dans la borne.

\section{Propriété intellectuelle}
La partie logicielle sera déposée sous licence GPL et accessible sur un dépôt git en libre accès. Les intervenants qui participeront au développement de ce projet seront crédités dans les fichiers de licence et le code source. La propriété intellectuelle leur reviendra.\\~

Rappel concernant la licence GPL : la licence GPL est la licence utilisée pour les logiciels libres. Si le but n'est pas commercial, le code source de ces logiciels peut être téléchargé, utilisé et modifié par tout le monde sans avoir besoin d'obtenir une autorisation du propriétaire. Le projet original doit tout de même être crédité dans la version modifiée ou dans le logiciel utilisant des morceaux du projet original.

\section{Budget}
Aucun budget n'est prévu par l'IUT pour ce projet. Les développements se feront dans le cadre de projets pédagogiques et scolaires ou dans le cadre de l'association JrCanDev.

Toutefois, le Sas Coluche et le port Boulogne Calais devront prendre un stagiaire de l'IUT pour la finalisation du travail sur place.

\section{Planification}
\subsection{Générale}
\begin{itemize}
\item \textbf{Septembre/octobre 2023} : Développement de l'application existante sous forme d'application fixe non évolutive par les étudiants de l'IUT dans le cadre de projet scolaire.
\item \textbf{D'octobre2023 à mars 2024} : Prototypage du logiciel de génération d'application tablette par les étudiants de l'IUT dans le cadre de projets scolaires.
\item \textbf{De mars 2024 à Juin 2024} : Finalisation du logiciel, création des documentations, création de l'application tablette avec ce logiciel dans le cadre d'un stage au port de Boulogne Calais. Possibilité de demander au stagiaire de configurer l'ordinateur de la borne.
\item \textbf{Été 2024} : Phase de test.
\item \textbf{Rentrée 2024} : Inauguration et photo dans le journal \smiley{}
\end{itemize}~\\

Bien que de la documentation sera fournie avec le logiciel, l'IUT s'engage à intervenir en cas de difficulté avec le logiciel produit même après l'inauguration.

\subsection{Découpage chronologique du développement du logiciel}
Le développemet du logiciel peut être découpé en tâches générales qui pourront être traitées de manière chronologique comme ceci :
\begin{enumerate}
\item Lister des modèles de pages possibles, les maquetter et les créer.
\item Développer l'application tablette existante à partir de ces modèles.
\item Réfléchir à une solution de stockage des modèles et des données.
\item Réfléchir à l'interface graphique utilisateur.
\item Réfléchir à l'architecture logicielle.
\item Développer le logiciel.
\item Tester et revenir au 6 tant que c'est nécessaire.
\end{enumerate}

\section{Conseils / Questions}
\begin{itemize}
\item Prévoir un ordinateur classique pour la borne avec un SSD pour un démarrage plus rapide.
\item Prévoir un accès facile à l'ordinateur dans la borne pour le transfert des données.
\item Prévoir un écran tactile FULL HD (1920x1080).
\item Configurer l'ordianteur pour qu'il démarre en mode tablette pour l'utilisation du clavier virtuel - Attention ! La procédure n'est pas la même sous windows 10, windows 11 ou linux.
\item Configurer l'ordinateur pour qu'il démarre en ouvrant la première page de l'application tablette en mode plein écran.
\item Qui commande le matériel informatique ?
\item Qui intégre le matériel dans la borne ?
\item Qui configure l'ordinateur de la borne ?
\end{itemize}

\end{document}

